\documentclass[11pt,a4paper]{jsarticle}

\usepackage{amsmath,amssymb}

\begin{document}

$\gamma_j$をボラティリティの減衰係数とする.
$\gamma_j$は区分的に微分可能,広義単調減少関数であり,$t \leq T_{j-1}$で1,$t \geq T_{j}$で0を満たす.
\begin{align}
g_j (t)
& =
-\frac{d}{d t} \gamma_j (t), \nonumber \\
G_j (t, T)
& =
\int_t^T g_j (u) \, du = \gamma_j (t) - \gamma_j (T). \nonumber
\end{align}
$G_j (T_{j-1}, T_{j}) = 1$を満たす.

示したいことは以下.
\begin{align}
\int_t^T g_k (u) G_l (t, u) \, du
& =
\frac{1}{2} G_k (t, T) G_l (t, T). \nonumber
\end{align}

$k = l$の場合は簡単.Cheyetteの他の論文でもよくやる計算.
\begin{align}
\int_t^T g_k (u) G_k (t, u) \, du
& =
\frac{1}{2} \int_t^T \frac{\partial}{\partial u} \Bigl\{ G_k (t, u)^2 \Bigr\} \, du \nonumber \\
& =
\frac{1}{2} G_k (t, T)^2. \nonumber
\end{align}

$k \neq l$の場合が謎.
\begin{align}
\int_t^T g_k (u) G_l (t, u) \, du
& =
\int_t^T g_k (u) \bigl\{ \gamma_l (t) - \gamma_l (u) \bigr\} \, du \nonumber \\
& =
G_k (t, T) \gamma_l (t) - \int_t^T g_k (u) \gamma_l (u) \, du. \nonumber
\end{align}

$k > l$とする.
$[T_{k-1}, T_{k}]$で$\gamma_l$は0なので
\begin{align}
\int_t^T g_k (u) G_l (t, u) \, du
& =
G_k (t, T) \gamma_l (t). \nonumber
\end{align}

$k < l$とする.
$[T_{k-1}, T_{k}]$で$\gamma_l$は1なので
\begin{align}
\int_t^T g_k (u) G_l (t, u) \, du
& =
G_k (t, T) \gamma_l (t) - G_k (t, T). \nonumber
\end{align}

$t$や$T$の位置関係についてさらに場合分けが必要そうだが,これらが$1/2 \, G_k (t, T) G_l (t, T)$に等しいように思えない.
少なくとも積分値が$k, l$について対称でないと2次形式の形にならないので後の議論が破綻してしまう.

\end{document}
